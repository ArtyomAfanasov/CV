%-------------------------------------------------------------------------------
%                             ADDITIONAL PACKAGES
%-------------------------------------------------------------------------------
\documentclass[
	a4paper,
	% showframes,
	% vline=2.2em,
	% maincolor=cvgreen,
	% sidecolor=gray!50,
	% sectioncolor=red,
	% subsectioncolor=orange,
	% itemtextcolor=black!80,
	% sidebarwidth=0.4\paperwidth,
	% topbottommargin=0.03\paperheight,
	% leftrightmargin=20pt,
	% profilepicsize=4.5cm,
	% profilepicborderwidth=3.5pt,
	% profilepicstyle=profilecircle,
	% profilepiczoom=1.0,
	% profilepicxshift=0mm,
	% profilepicyshift=0mm,
	% profilepicrounding=1.0cm,
	% logowidth=4.5cm,
	% logospace=5mm,
	% logoposition=before,
]{fortysecondscv}


% improve word spacing and hyphenation
\usepackage{microtype}
\usepackage{ragged2e}

% uncomment in case you don't want any hyphenation
% \usepackage[none]{hyphenat}

% take care of proper font encoding
\ifxetexorluatex
	\usepackage{fontspec}
	\defaultfontfeatures{Ligatures=TeX}
	%	\newfontfamily\headingfont[Path = fonts/]{segoeuib.ttf} % local font
\else
	\usepackage[utf8]{inputenc}
	\usepackage[T1]{fontenc}
	%	\usepackage[sfdefault]{noto} % use noto google font
\fi

% enable mathematical syntax for some symbols like \varnothing
\usepackage{amssymb}

%-------------------------------------------------------------------------------
%                            PERSONAL INFORMATION
%-------------------------------------------------------------------------------
%% mandatory information
\cvname{Artyom Afanasov}
% \cvjobtitle{Panda Scientist,\\[0.2em] Panda of the Year}

%% optional information
\cvprofilepic{pics/profile.jpg}
% \cvlogopic{pics/profile.jpg}

\cvbirthday{January 3, 1999}
\cvaddress{Russia, Saint Petersburg, Peterhof} % Use \newline if more than 1 line is required
\cvcustomdata{\faGithub}{\href{https://github.com/ArtyomAfanasov}{ArtyomAfanasov}}
\cvmail{afanasov.artyom@gmail.com}
\cvcustomdata{\faTelegram}{\href{https://t.me/patoshca}{@patoshca}}
\cvphone{+7(981)718-54-81}

% personal website
% \cvsite{https://pandascience.net}

% pgp key
% \cvkey{4096R/FF00FF00}{0xAABBCCDDFF00FF00}

%-------------------------------------------------------------------------------
%                              SIDEBAR 1st PAGE
%-------------------------------------------------------------------------------
% add more profile sections to sidebar on first page
\addtofrontsidebar{
	% include gosquare national flags from https://github.com/gosquared/flags;
	% naming according to ISO 3166-1 alpha-2 country codes
	\graphicspath{{pics/flags/}}

	% % social network accounts incl. proper hyperlinks
	% \profilesection{Social Network}
	% 	\begin{icontable}{2.5em}{1em}
	% 		\social{\aiOverleafSquare}
	% 			{https://de.overleaf.com/latex/templates/forty-seconds-cv/pztcktmyngsk}
	% 			{Overleaf Template Link}
	% 		\social{\faGithub}
	% 			{https://github.com/PandaScience/FortySecondsCV}
	% 			{Github Project Page Link}
	% 	\end{icontable}

	\profilesection{Languages}
	\pointskill{\flag{RU.png}}{Russian}{5}
	\pointskill{\flag{GB.png}}{English}{3}

}

%-------------------------------------------------------------------------------
%                              SIDEBAR 2nd PAGE
%-------------------------------------------------------------------------------
\addtobacksidebar{
	\profilesection{Skills}
    \pointskill{\faWindows}{.NET}{3}[4]
	\skill[1.8em]{\faCaretRight}{C\#}
	\skill[3.0em]{\faCaretRight\faCaretRight}{MSTest}
	\skill[3.0em]{\faCaretRight\faCaretRight}{TPL}
	\skill[3.0em]{\faCaretRight\faCaretRight}{DI}  
    \skill[3.0em]{\faCaretRight\faCaretRight}{Reflection}
    \skill[3.0em]{\faCaretRight\faCaretRight}{WPF}
    \skill[3.0em]{\faCaretRight\faCaretRight}{Exception Handling} 
    \skill[3.0em]{\faCaretRight\faCaretRight}{CodeStyle}
	\skill[1.8em]{\faCaretRight}{F\#}
	\pointskill{\faCaretRight}{Java}{1}[4]
	\pointskill{\faCaretRight}{TypeScript}{1}[4]
	\pointskill{\faCaretRight}{VCS}{3}[4]
	\skill[1.8em]{\faCaretRight}{git, TortoiseGit, GitLub, GitHub}
	\pointskill{\faDatabase}{Databases}{2}[4]
	\skill[1.8em]{\faCaretRight}{MsSQL, ORM}
	\skill[1.8em]{\faCaretRight}{Amazon RDS}
	\skill[1.8em]{\faCaretRight}{Microsoft Azure SQL Databases}
	\pointskill{\faCogs}{CI/CD}{2}[4]
	\skill[1.8em]{\faCaretRight}{TeamCity, Docker}
	\pointskill{\faCaretRight}{Software Design}{2}[4]
	\skill[1.8em]{\faCaretRight}{UML-diagrams}
	\pointskill{\faCaretRight}{Sniffing}{2}[4]
	\skill[1.8em]{\faCaretRight}{Wireshark, Fiddler, Burp Suite}
	\pointskill{\faCaretRight}{Assembler}{2}[4]
	\skill[1.8em]{\faCaretRight}{Intel x86}
	\skill[1.8em]{\faCaretRight}{DSP C66x}
	\pointskill{\faLinux}{Linux}{2}[4]
	\skill[1.8em]{\faCaretRight}{bash tools, VM administration}
	\pointskill{\faCaretRight}{Jupyter-notebook}{2}[4]
	\skill[1.8em]{\faCaretRight}{python}
	\skill[1.8em]{\faCaretRight}{pretty result via Markdown}
	\pointskill{\faCaretRight}{Cloud computing}{2}[4]
	\skill[1.8em]{\faCaretRight}{AWS}
	\skill[1.8em]{\faCaretRight}{Microsoft Azure}
	\pointskill{\faCaretRight}{Information Security}{1}[4]
	\skill[1.8em]{\faCaretRight}{exploit tools, bash}
	\skill[1.8em]{\faCaretRight}{virtual machine, network}
	\pointskill{\faCaretRight}{Jira}{3}[4]
    \profilesection{About Me}
	\aboutme{
        \faHeart{} IT, volleyball, calisthenic, piano, guitar
	}	





	% \aboutme{
	% 	The giant panda is a terrestrial animal and primarily spends its life
	% 	roaming and feeding in the bamboo forests of the Qinling Mountains and in
	% 	the hilly province of Sichuan.
	% }

	% \profilesection{Diagrams}
	% \begin{sidebarminipage}
	% 	\chartlabel{Bubble}
	% 	\chartlabel{Diagrams}
	% 	\chartlabel{with}
	% 	\chartlabel{proper}
	% 	\chartlabel{overflow}
	% 	\chartlabel{protection}
	% 	\chartlabel{for}
	% 	\chartlabel{labels}
	% \end{sidebarminipage}

	% \begin{figure}\centering
	% 	\smartdiagram[bubble diagram]{
	% 		\textcolor{white}{\textbf{Being a}} \\
	% 		\textcolor{white}{\textbf{Panda}}, % center bubble
	% 		\textcolor{black!90}{Eating},
	% 		\textcolor{black!90}{Sleeping},
	% 		\textcolor{black!90}{Rolling},
	% 		\textcolor{black!90}{Playing},
	% 		\textcolor{black!90}{Chilling}
	% 	}
	% \end{figure}

	% \chartlabel{Wheel Chart}

	% \wheelchart{3.7em}{2em}{%
	% 20/3em/maincolor!50/Chill,
	% 15/3em/maincolor!15/Play,
	% 30/4em/maincolor!40/Sleep,
	% 20/3em/maincolor!20/Eat
	% }

	% \profilesection{Barskills}
	% \barskill{\faSkyatlas}{Wearing asian rice hats}{60}
	% \barskill{\faImage}{Playing Chess}{30}
	% \barskill{\faMusic}{Playing the bamboo flute}{50}

	% \profilesection{Memberships}
	% \begin{memberships}
	% 	\membership[4em]{pics/logo.png}{PandaScience.net}
	% 	\membership[4em]{pics/logo.png}{Some longer text spanning over more than
	% 		only one line}
	% \end{memberships}
}


%-------------------------------------------------------------------------------
%                         TABLE ENTRIES RIGHT COLUMN
%-------------------------------------------------------------------------------
\begin{document}

\par
\makefrontsidebar{}

\cvsection{Working Experience}
\begin{cvtable}[1.5]
  \cvitem{03.2021 -- now}{Software Engineer}{Belkasoft} \\

  \cvitem{09.2019 -- 12.2019}{Junior .NET Developer}{KORUS Consulting CIS Ltd.}
    {\\ \emph{ESPHERE Courier} \\
     Worked with Electronic Data Interchange (\colorbox{cvsidecolor}{EDI}) on the main company project Esphere Courier (\colorbox{cvsidecolor}{REST API}), fixing bugs, adding new functionality and refactoring web services. For example:
     \begin{itemize}
         \item Added processing of electronic documents and checking their validity for a specific organization
         \item Fixed incorrect work with export to Excel
         \item I myself was founding places in the existing project logic that could be improved, and was carrying out refactoring
     \end{itemize}
     Development was carried out in \colorbox{cvsidecolor}{C\#}, \colorbox{cvsidecolor}{TypeScript}.
    }

\end{cvtable}

\cvsection{Internships}
\begin{cvtable}[1.5]

	\cvitem{07.2019 -- 08.2019}{.NET Developer (Intern)}{KORUS Consulting CIS Ltd.}
	{\\ \emph{Project for working with electronic signatures} \\
    Added functionality to enhance a electronic signature in a company project, using \colorbox{cvsidecolor}{C\#}, \colorbox{cvsidecolor}{C library},  \colorbox{cvsidecolor}{marshalling}. Development was \\ carried out through \colorbox{cvsidecolor}{TDD}. \\ \\
     \emph{Service Integration} \\
     Improved the interaction of a company project with the Jivosite API by adding processing of requests and responses from Jivosite API, using \colorbox{cvsidecolor}{MassTransit}, as well as their serialization and deserialization. The architecture for adding related functionality has been improved with \colorbox{cvsidecolor}{reflection} and \colorbox{cvsidecolor}{Dependency Injection}.
    }

\end{cvtable}


\cvsection{Education}

\cvsubsection{Study}
\begin{cvtable}[1.5]
	\cvitem{2017 -- now}{Bachelor Studies}{Saint Petersburg State University}
	{Software and Administration of Information Systems,
		Department of Software Programming. \\
	}
	\cvitem{2006 -- 2017}{Secondary education}{Gymnasium named after A. Green of the city of Kirov}
	{ * Graduated with a gold medal.}
\end{cvtable}

\cvsection{Online courses}
\begin{cvtable}[1.5]
	\cvitem{2021 -- now}{{Functional programming via Haskell}}{Computer Science Center}
	{Main concepts of functional programming and Haskell.}
	\cvitem{2020 -- now}{{Introduction to Linux}}{Bioinformatics Institute}
	{Basic concepts of Linux.}
\end{cvtable}

\newpage
\makebacksidebar


% \cvsection{Publications}
% \begin{cvtable}
% 	\cvpubitem{Cooking: 100 recipes for lazy Pandas}{Me and My Panda Friends}
% 		{Panda's Culinary World}{2010}
% 	\cvpubitem{Pandastasia}{Still Me}{Bamboo Books Assoc.}{2005}
% \end{cvtable}

% \newgeometry{
% 	top=\topbottommargin,
% 	bottom=\topbottommargin,
% 	right=\leftrightmargin,
% 	left=\leftrightmargin
% }

\addtobacksidebar{}

%\newpage
%\begin{sidebar}
%    \nameandjob%

%    \setlength{\parskip}{1ex}

%    \profilesection{About Me}
%	\aboutme{
%        \faHeart{} programming
%	}
%\end{sidebar}

\cvsection{Projects}
\begin{cvtable}[1.5]
	\cvitem{2020 -- now}{{Graduation work. Digital forensics application.}}{}
	{Closed source project.}
	\cvitem{2019 -- 2020}{{Term paper. Implementing Asymmetric Marker processing on the C66x DSP.}}{}
	{\\ AMP (10.17587/prin.9.156-162) implementation on a specialized DSP C66x processor for communication with ARM. I have studied the architecture of the system on a chip \colorbox{cvsidecolor}{EVMK2H}, interaction with SoC through \colorbox{cvsidecolor}{Code Composer Studio}, the architecture of \colorbox{cvsidecolor}{C66x DSP} processor and \colorbox{cvsidecolor}{assembly language DSP}. And then I have implemented the layers of the AMP model in assembly language DSP. And my assembly language implementation turned out to be 1.7 times faster than the C implementation with the -O3 optimization.}
	\cvitem{2019}{{Term paper. CI/CD pipeline configuration for a microservice architecture web application.}}{}
	{During my term paper on the configuration of the \colorbox{cvsidecolor}{CI/CD} pipeline for the microservice architecture web application (my role in the project was DevOps) I have automated the entire pipeline (from committing to GitHub to running a microservice in the virual machine): commit, testing, building a docker-image, pushing the docker-image to DockerHub, connecting to a VM and creating a container. In this work I have used:
    \begin{itemize}
    \item \colorbox{cvsidecolor}{Linux VM machines} \colorbox{cvsidecolor}{AWS} and \colorbox{cvsidecolor}{Microsoft Azure} for hosting and database services
    \item \colorbox{cvsidecolor}{TeamCity} for pipeline configuration
    \item \colorbox{cvsidecolor}{Docker} for flexible delivery.
    \end{itemize}}
	\cvitem{2018}{{Term paper. Small computer multiplayer game.}}{}
	{A computer game that supports multiplayer. And as a developer, I do not need to set up a game server. Each player can be a server, thanks to \colorbox{cvsidecolor}{Photon Unity Networking}, therefore people can play anytime. \colorbox{cvsidecolor}{DiffMerge} was used to prevent merge conflicts. \colorbox{cvsidecolor}{Unity} was a game editor.}
	\cvitem{2018}{{Summer SPBU project. Neurointerface for computer control.}}{}
	{I have received data of electronic activity of the brain (P300 wave) using \colorbox{cvsidecolor}{the EMOTIV EPOC neurointerface} and \colorbox{cvsidecolor}{SDK for neurointerface} for \colorbox{cvsidecolor}{C\#}.}
\end{cvtable}

\cvsignature

\end{document}
